\documentclass[11pt]{article}
\special{papersize=8.5in,11in}
\usepackage{libertine}
\usepackage{geometry}
%\usepackage[hidelinks]{hyperref}
\usepackage{hyperref}

\title{Research Statement}
\author{Shih-Kai Lin}
\date{}

\begin{document}
\maketitle

My research interest is probing physics beyond the Standard Model at the intensity frontier. The Standard Model, since its development in the 1970s, has so successfully explained \emph{almost} all experimental results and predicted new phenomena in fundamental physics. However physicists know the Standard Model cannot be the ultimate theory of everything, not only because of the theoretical reason that physicists haven't incorporated the gravitational force into quantum theory successfully, but also the empirical reason that the Standard Model still leaves some phenomena inexplicable. Among those inexplicable phenomena are: dark matter and dark energy, neutrino masses and matter/antimatter asymmetry. The way to understand more about our material world beyond our current knowledge is through experiments. I work on these experiments in the hope of contributing to human understanding of the mother nature which may some day lead to a complete theory of the material world.

\paragraph*{Previous Research}

Joining Taiwan EXperiment On NeutrinO (TEXONO) group was my first step to explore physics beyond the Standard Model. The TEXONO group is known for its record-setting measurement of the neutrino magnetic moment~\cite{HBL2003}. If neutrinos have mass, they will have magnetic moment and can in turn scatter off of the electrons by magnetic scattering. By measuring the electron recoil spectrum, the neutrino magnetic moment can be obtained. TEXONO uses high-purity germanium (HPGe) detectors to measure the electron recoil which have the advantages like very high energy resolution and very low energy threshold. The precision recoil spectrum measurement ultimately relies on the understanding of the energy scale of the HPGe. I was interested in the HPGe energy calibration, energy resolution and detector response to different particles. I set up the energy scale with all kinds of calibration sources such as $^{55}$Fe, $^{137}$Cs and $^{60}$Co. Since the magnetic scattering dominates at low energy, I also tried to lower the detector threshold with pulse shape discrimination technique. In the mean time I was involved in analyzing the data of recoil energy spectrum which led to an updated neutrino magnetic moment measurement~\cite{HTW2007}. The detectors and shielding in the nuclear power plant underwent upgrade several times. Each time I was actively involved in the installation and assured the low background level of TEXONO experiments. Through a very careful background study, the energy spectrum measured with an upgraded ultra-low-energy germanium detector was used to obtain a contour in the dark matter exclusion plot~\cite{STL2009}. Having gained experience in neutrino physics and dark matter search, I felt urged to move on.

\paragraph*{Current Research}

One of the biggest scientific achievements in the last decade was the discovery of neutrino oscillation which solved the long-standing solar neutrino problem. Since then, understanding how much different neutrinos mix has been the major task of neutrino physicists. How much neutrinos mix can be quantified by three mixing angles $\theta_{12}$, $\theta_{23}$ and $\theta_{13}$. By the end of 2008, $\theta_{13}$ was the only unknown parameter of the three. Measuring $\theta_{13}$ is not only to complete the mixing picture but may have far-reaching consequences in the matter/antimatter asymmetry. I felt excited to join the University of Houston (UH) high energy group in 2009 working on the Daya Bay reactor neutrino experiment whose scientific goal is to do the most precise measurement of $\theta_{13}$. At the time Daya Bay was designed, there were indications that $\theta_{13}$ could be small~\cite{CHOOZ1998}. Daya Bay therefore carefully designed a muon system with a muon detection efficiency as high as 99.5\%. The Daya Bay muon system consists of a water Cherenkov pool in which the antineutrino detectors are submerged and Resistive Plate Chamber (RPC) modules covering the top of the pool. RPC is a parallel plate charged particle gas detector known to be simple in design, cheap to manufacture and easy to scale up to large area with high spacial resolution. Our group's interest is in the RPC and I was largely involved in the RPC bare chamber QC/QA~\cite{LHM2011} and RPC module assembly~\cite{JLX2011}. During Daya Bay installation, I participated in all kinds of work in the muon system such as water pool Photomultiplier Tube (PMT) assembly, cable routing and cabling, and RPC module installation. After Daya Bay started data taking, I independently analyzed the RPC data to obtain a \textit{in situ} measurement of the RPC efficiency and the underground muon flux~\cite{ZN2013}. Now I am working on the muon induced neutrons which is an important background for sensitive underground experiments and my analysis will serve as one of the independent analyses for the Daya Bay's coming paper on muon backgrounds. In April 2012, only 4 months since Daya Bay's first data taking, Daya Bay collaboration published the first result on the nonzero $\theta_{13}$ with a significance of 5.2$\sigma$~\cite{DYB2012}.

\paragraph*{Future Research}

$\theta_{13}$ turns out to be as large as neutrino physicists dream about. Its largeness makes next steps relatively easier~\cite{HM2011}. The immediate prominent questions to answer are of course the remaining parameter in the mixing matrix, the CP-violating phase $\delta_{CP}$, whether neutrinos are Majorana particles, and the mass hierarchy of neutrinos. In view of the versatility of NO$\nu$A's physics goals which aim to answer 2 of the prominent questions, NO$\nu$A would be one of my best choices to continue my journey to another discovery of physics beyond the Standard Model. Successfully measuring the $\delta_{CP}$ may resolve one of the biggest mysteries in the universe: why do we exist without being annihilated by antimatter? My interest in muon physics also makes me open to another intensity frontier experiment, muon g-2. For the experiments still under construction, I can bring my attentive hardware experience to help on installation, commissioning and trouble shooting. For data analysis, I am interested in detector simulation with Geant4, event reconstruction and particle identification with machine learning techniques. This is an exciting era for particle physicists and I believe I can make contributions to the imminent big discoveries.

\bibliographystyle{ieeetr}
\bibliography{ResearchStatementRef}


\end{document}