\documentclass[10pt]{article}
\special{papersize=8.5in,11in}
\usepackage[latin1]{inputenc}
\usepackage{libertine}
\usepackage{amsmath}
\usepackage{amsfonts}
\usepackage{amssymb}
\usepackage{geometry}

\title{Personal Statement}
\date{April 7, 2014}
\author{Shih-Kai Lin}

\begin{document}
\maketitle

%\begin{itemize}
    %\item I was interested in mathematics for as long as I can remember
    %\item I became interested in computer science in my junior year of high school
    %\item I became interested in big data (handling it, analyzing it, processing it) in the summer after sophomore year. I was interning at a big data company (Explorys) and it showed me a new class of very difficult problems which were able to provide really valuable insights and help make the world better, just by using computer science and mathematics (statistics).
    %\item I did research with Dr. Ruoming Jin (computer science) on two major projects: creating a distributed virtual disk (to handle huge amounts of raw data easily) and creating a distributed graph database (to handle large amounts of normally unstructured data in a natural way, since a wealth of data can be represented as graphs). This got me interested in a lot of the problems associated with huge amounts of data.
    %\item In my work at GraphSQL, my interest in data analysis has intensified. I feel ill-equipped to do a lot of tasks (statistics, numerical algorithms for computing, etc.).
    %\item I want to pursue a master's degree in applied math to get a better understanding of statistics and computational mathematics so that I can be more effective at handling and analyzing big data.
%\end{itemize}

%For as long as I can remember, my passion has been solving hard problems. Sometimes this has manifested in a passion for mathematics, or computer science, or logic puzzles, but the constant thread is the same: I enjoy solving difficult and important problems. I have been interested in computer science, and consequently computational mathematics, since my junior year of high school. I first took a programming class to help with my study of mathematics, but I fell in love with programming and computational mathematics, using the two to create a real-time traction control system which leveraged real-time approximations of integrals.

I have been enthralled by mathematics and science since I started school. When I was in high school, I would spend hours struggling with a hard problem in math or physics because for me, the more racking the course is, the more rewarding the answer will be.

%In the summer of 2011, I became interested in Big Data. Loosely defined, Big Data is the field concerned with handling, processing, and analyzing datasets which are so large that traditional methods will fail. In that summer, I was a software engineering intern at Explorys, a Big Data healthcare analytics company in Cleveland. My internship introduced me to a whole new class of extremely difficult problems which have a significant real-world impact. It inspired me to see computer science, mathematics, and statistics having a significant real-world impact, solving life-or-death problems.

In the fall of 1996, I entered National Taiwan University majoring in Computer Science, which is in my opinion a kind of applied mathematics and deals with practical problems pure mathematicians usually leave aside. During the time I still maintained a high degree of interest in physics, attended physics colloquiums and took core courses in physics curriculum. The training in computer science made me good at solving real world problems with computers and a quick learner of new computer tools for solving problems.

%In the spring and summer of 2012, I did computer science research with Dr. Ruoming Jin on two major projects: creating a distributed virtual disk to handle huge amounts of raw data easily; and creating a distributed graph database to handle large amounts of network data in a natural way. Both of these projects were Big Data projects and increased my interest in the Big Data field by giving me experience with portions of it I had not seen before.

After graduation and 2 years of mandatory military service, I entered the department of physics at National Taiwan University for graduate study. At the time I was open to interesting problems in physics and low temperature experiments and the electron transport properties of new materials seemed exciting to me. Therefore I joined the semiconductor group, learned techniques in low temperature experiments, measured the transport properties of In$_x$Ga$_{1-x}$N and published a paper on it. My open-mindedness also led me to learn of Taiwan Experiment on Neutrino(TEXONO) group through my classmate. After attending a talk by Dr. Henry Wong, I decided to join his group after my Master's degree to study the intriguing, elusive particle, neutrino.

%In January, 2013, I began started working as a software engineer at GraphSQL. At GraphSQL, we are developing an extremely high-performance scalable graph storage engine and graph processing engine to allow graph analytics on extremely large graphs. As an example, our system is able to handle a 500 million vertex, many-billion edge graph on a single high-end server. My own role is developing cutting-edge graph analytics on top of our platform. I have implemented link recommendation functions and centrality measure approximation functions. In this role, I have seen the importance of computational mathematics and statistics to the development of high-performance cutting-edge algorithms and analytics.

In TEXONO, I acquired comprehensive experience from hardware to software analysis including routine detector cryogenic state maintenance, detector energy calibration with radioactive sources, DAQ, operation of NIM modules and data analysis. My interest in particle physics was ever growing and I started to yearn for a more advanced study in the US. Right in the summer of 2008, at the Overseas Chinese Physicist Association Underground Science Conference at the University of Hong Kong, I met my adviser, Kwong, who needed a graduate student to work on Daya Bay. I immediately seized the chance and entered the University of Houston for my Ph.D. study since 2009.

%I wish to pursue a master's degree in applied mathematics at Kent State so that I can become more effective at handling and analyzing problems in the Big Data field. I feel that I would benefit immensely from a strong background in statistics and numerical analysis. This program would allow me to become a far more effective software engineer and data scientist so that I can help solve the world's difficult problems.

At University of Houston, our main interest is in RPC. I participated in RPC bench tests like detector response to different particles and gas composition. I also participated in RPC bare chamber QC/QA in Beijing. During Daya Bay installation, I participated in MACRO PMT pressure test, AD PMT transportation, RPC module installation, water pool PMT assembly, cable routing and cabling. Now Daya Bay is a major accomplishment in the history of neutrino physics and I feel honored to have worked on it!

It is time for me to move on to the next stage of my career. Physics, especially particle physics, is still my favorite subject. I will certainly pursue postdoc opportunities in closely related experiments. However this is not to say I rule out other possibilities. I am still fond of applying the fundamental principles to real world problems. I believe the training I acquired in my education makes me capable of making contributions to other scientific fields where mathematical modeling, computer algorithms or physics principles are needed.

\end{document}