\documentclass[11pt,letterpaper]{moderncv}
\usepackage{libertine}    %% or newtxtext for times 
\moderncvstyle{classic}
\moderncvcolor{green}
\usepackage[scale=0.75]{geometry}

\name{Shih-Kai}{Lin}
\title{Research Assistant}
\address{2121 El Paseo St Apt 1307}{Houston, TX, 77054}
\phone[mobile]{+1~(832)~907~9035}
%\phone[fixed]{+2~(345)~678~901}
%\phone[fax]{+3~(456)~789~012}
\email{shihkailin78@gmail.com}
%\homepage{www.johndoe.com}
\social[linkedin]{shihkailin}
%\social[twitter]{jdoe}
%\social[github]{jdoe}
%\extrainfo{additional information}
%\photo[64pt][0.4pt]{picture}
\quote{}


\begin{document}
\makecvtitle

\section{Education}
%\cventry{year--year}{Degree}{Institution}{City}{\textit{Grade}}{Description}
\cventry{2009--present}{Doctor of Philosophy in Physics}{University of Houston}{Houston, TX}{\textit{GPA: 3.867/4.0}}{}
\cventry{2003--2005}{Master of Science in Physics}{National Taiwan University}{Taipei, Taiwan}{}{}
\cventry{1996--2001}{Bachelor of Science in Computer Science}{National Taiwan University}{Taipei, Taiwan}{}{}


\section{Doctoral Dissertation}
\cvitem{title}{\emph{Neutron Production by Cosmic Ray Muons}}
\cvitem{supervisors}{Kwong Lau}
\cvitem{description}{Neutrons produced by cosmic ray muons are a background of underground experiments. Understanding neutron yield is particularly important for future low background experiments. In this work mechanisms of neutron production by muons are discussed and neutron yield is measured with data from Daya Bay Reactor Neutrino Experiment.}

\section{Experience}
\subsection{Vocational}
\cventry{2009--present}{Research Assistant}{University of Houston}{Houston, TX}{}{Member of particle physics group and Daya Bay Reactor Neutrino Experiment. Actively participated in the muon detector system QC/QA, installation, commissioning, operation and data analysis.%\newline{}%
%Detailed achievements:%
\begin{itemize}%
	\item	Analyze data for neutron production by cosmic ray muons.
  \item Developed a reconstruction algorithm to reconstruct the incident position and the track of cosmic ray muons.
	\item Designed and built a small resistive plate chamber for monitoring the gas composition.
  \item Experience in operating radioactive sources such as $^{137}$Cs, $^{90}$Sr and $^{60}$Co for resistive plate chamber response study.
  \item Designed and built a collimator for shielding and collimating $^{60}$Co.
  \item Monte Carlo simulation on photoelectron production by photons and electron transport in Bakelite.
  \item Developed a web based 3D muon event display with WebGL and MySQL technology.
  \item During hardware installation, led a team of technicians to route 400 electronic cables from a 10m deep pool to the electronics control room in a week.
\end{itemize}}
\cventry{2005--2008}{Research Assistant}{Academia Sinica}{Taipei, Taiwan}{}{Joined Taiwan Experiment on Neutrino(TEXONO) group and worked on high purity Germanium detector response to different particles, energy calibration and data analysis.%\newline{}%
%Detailed achievements:%
\begin{itemize}%
	\item Analyzed data used for dark matter exclusion plots.
	\item Calibrated the Ge detector with $^{137}$Cs, $^{60}$Co, $^{55}$Fe, $^{32}$P, etc.
	\item Maintained the Ge detector located 28m away from the core of No.1 reactor of Kuo-Sheng Nuclear Power Station.
  \item Utilizing pulse shape discrimination(PSD) technique to lower detector threshold.
  \item Developed an algorithm utilizing Fast Fourier Transform spectra to reject near-threshold detector noise.
\end{itemize}}
\cventry{2003--2005}{Research Assistant}{National Taiwan University}{Taipei, Taiwan}{}{Joined semiconductor group and worked on electron transport properties of Indium Gallium Nitride.%\newline{}%
%Detailed achievements:%
\begin{itemize}%
  \item Measured and published a paper on the transport properties of Indium Gallium Nitride.
  \item Extensive experience in high vacuum and low temperature experiments.
\end{itemize}}
\cventry{2001--2003}{Lieutenant}{Mandatory Military Service}{}{}{Maintained and developed a windows application for database management of parts of radio equipment with Borland Delphi.\newline{}%
}
\subsection{Teaching}
\cventry{2004}{Teaching Assistant}{National Taiwan University}{Taipei, Taiwan}{}{Responsible for grading, solving and answering questions about University Physics homework problems for about 50 students.}


\section{Honors and Professional Memberships}
\cvitem{Mar 2013}{American Physical Society}
\cvitem{Nov 2009}{Golden Key International Honour Society}
\cvitem{2005}{Government Fellowship for Studying Abroad from Ministry of Education, Taiwan}


\section{Conference Talks}
\cvitem{Apr 2013}{Shih-Kai Lin, ``A Geometric Method for Measuring Muon Induced Neutrons at Daya Bay", APS April Meeting 2013}


\section{Languages}
\cvitem{Chinese}{Native}
\cvitem{English}{Fluent}


\section{Computer Skills}

\cvitem{Language}{\textsc{C/C++}, \textsc{MS Visual C++ 4.0}, \textsc{Borland Delphi}, \textsc{HTML}, \textsc{Python}, \textsc{Bash}, SQL, \textsc{JavaScript}, PHP, AJAX, \LaTeX}

\cvitem{Software}{\textsc{MySQL}, \textsc{SQLite}, \textsc{phpMyAdmin}, \textsc{Apache}, \textsc{Squid}, \textsc{Origin}, MATLAB}
\cvitem{System}{\textsc{Windows}, \textsc{Linux} (both \textsc{Debian} and \textsc{Red Hat} based)}


\section{Publications}
\cvitem{2013}{Z. Ning, Q.M. Zhang, J.L. Xu, L. Lebanowski, J.W. Zhang, C.G. Yang, M. He, J. Zhao, J.H. Zou, V. P{\v e}{\v c}, \underline{\textbf{\emph{Sh.-K. Lin}}}, M.Y. Guan, H.F. Hao, L. Zheng, X.L. Ji, F. Li, K. Lau and V. Vorobel, ``Calibration algorithms of RPC detectors at Daya Bay Neutrino Experiment", \textit{JINST} 8 (2013) T03007}

\cvitem{2013}{(Daya Bay Collaboration), “Improved measurement of electron antineutrino disappearance at Daya Bay”, \textit{Chinese Phys. C} Vol. 37, No. 1 (2013) 011001}

\cvitem{2012}{(Daya Bay Collaboration), “Observation of electron-antineutrino disappearance at Daya
Bay”, \textit{Phys. Rev. Lett.} 108, 171803 (2012)}
\cvitem{2012}{(Daya Bay Collaboration), “A side-by-side comparison of Daya Bay antineutrino
detectors”, \textit{Nucl. Instr. Meth A} 685, 78-97 (2012)}

\cvitem{2011}{XU Ji-Lei, GUAN Meng-Yun, YANG Chang-Gen, WANG Yi-Fang, ZHANG Jia-Wen, LU Chang-Guo, Kirk McDonald, Robert Hackenburg, Kwong Lau, Logan Lebanowski, Cullen Newsom, \underline{\textbf{\emph{Lin Shih-Kai}}}, Jonathan Link, MA Lie-Hua, Viktor P{\v e}{\v c}, Vit Vorobel, CHEN Jin, LIU Jin-Chang, ZHOU Yong-Zhao, LIANG Hao, ``Design and preliminary test results of Daya Bay RPC modules", \textit{Chinese Phys. C} 35(9): 844-850}

\cvitem{2011}{Liehua Ma, Logan Lebanowski, Jin Chen, Mengyun Guan, Robert Hackenburg, Kwong
Lau, \underline{\textbf{\emph{Shih-Kai Lin}}}, Changguo Lu, Kirk McDonald, Cullen Newsom, Zhe Ning, Viktor Pec,
Sen Qian, Vit Vorobel, Yifang Wang, Jilei Xu, Changgen Yang, Jiawen Zhang, Qingmin
Zhang, “The mass production and quality control of RPCs for the Daya Bay experiment”,
\textit{Nuclear Instruments and Methods in Physics Research A} 659 (2011) 154–160}

\cvitem{2010}{M. Deniz, S.T. Lin, V. Singh, J. Li, H.T. Wong, S. Bilmis, C.Y. Chang, H.M. Chang, W.C. Chang, C.P. Chen, M.H. Chou, K.J. Dong, J.M. Fang, C.H. Hu, G.C. Jon, W.S. Kuo, W.P. Lai, F.S. Lee, S.C. Lee, H.B. Li, H.Y. Liao, C.W. Lin, F.K. Lin, \underline{\textbf{\emph{S.K. Lin}}}, Y. Liu, J.F. Qiu, M. Serin, H.Y. Sheng, L. Singh, R.F. Su, W.S. Tong, J.J. Wang, P.L. Wang, S.C. Wu, S.W. Yang, C.X. Yu, Q. Yue, M. Zeyrek, D.X. Zhao, Z.Y. Zhou, Y.F. Zhu, and B.A. Zhuang, ``Measurement of $\bar{\nu}_e$-elelctron cross section with a CsI(Tl) scintillating crystal array at the Kuo-Sheng nuclear power reactor", \textit{Phys. Rev. D} 81, 072001 (2010)}

\cvitem{2009}{S.T. Lin, H.B. Li, X. Li, \underline{\textbf{\emph{S.K. Lin}}}, H.T. Wong, M. Deniz, B.B. Fang, D. He, J. Li, C.W. Lin, F.K.
Lin, V. Singh, X.C. Ruan, J.J. Wang, Y.R. Wang, S.C. Wu, Q. Yue, and Z.Y. Zhou, “New limits
on spin-independent and spin-dependent couplings of low-mass WIMP dark matter with
a germanium detector at a threshold of 220 eV”, \textit{Phys. Rev. D} 79:061101, 2009}

\cvitem{2008}{H Y Liao, H M Chang, M H Chou, M Deniz, H X Huang, F S Lee, H B Li, J Li, C W Lin, F K Lin, \underline{\textbf{\emph{S K Lin}}}, S T Lin, V Singh, H T Wong and S C Wu, ``Production and decay of the $^{73}$Ge*(1/2$^-$) metastable state in a low-background germanium detector", \textit{J. Phys. G: Nucl. Part. Phys.} 35 (2008) 077001}

\cvitem{2007}{H. T. Wong, H. B. Li, S. T. Lin, F. S. Lee, V. Singh, S. C. Wu, C.Y. Chang, H. M. Chang, C. P.
Chen, M. H. Chou, M. Deniz, J. M. Fang, C. H. Hu, H. X. Huang, G. C. Jon, W. S. Kuo, W. P.
Lai, S. C. Lee, J. Li, H.Y. Liao, F. K. Lin, \underline{\textbf{\emph{S. K. Lin}}}, J. Q. Lu, H.Y. Sheng, R. F. Su, W. S. Tong, B.
Xin, T. R. Yeh, Q. Yue, Z.Y. Zhou, and B. A. Zhuang, “Search of Neutrino Magnetic
Moments with a High-Purity Germanium Detector at the Kuo-Sheng Nuclear Power
Station”, \textit{Phys. Rev. D} 75:012001, 2007}
\cvitem{2005}{\underline{\textbf{\emph{Shih-Kai Lin}}}, Kun-Ta Wu, Chao-Ping Huang, C.-T. Liang, Y. H. Chang, Y. F. Chen, P. H.
Chang, N.C. Chen, H. C. Peng, C. F. Shih, K. S. Liu and T. Y. Lin, ``Electron Transport in In-rich In$_x$Ga$_{1-x}$N films", \textit{Journal of Applied Physics} 97, 046101 (2005)}

\end{document}