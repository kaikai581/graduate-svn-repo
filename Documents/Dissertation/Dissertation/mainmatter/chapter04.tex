\chapter{Neutron Production Mechanisms in Liquid Scintillator}

Muons interact with matter through exchange of virtual photons. The generated neutrons can be categorized into two kinds, directly generated neutrons and secondary neutrons~\cite{Malgin2008}. Directly generated neutrons can be visualized with the following picture. The electromagnetic fields generated by charged particles can be thought of as a swarm of virtual photons traveling with the particles. When a muon passes by an atomic nucleus, it interacts with the nucleus by exchanging virtual photons with the nucleus. Being struck by the virtual photons, the nucleus could become excited and when it deexcites, neutrons could be released. To quantify direct neutron generation, one can first get the virtual photon frequency spectrum with Williams-Weissacker's method of virtual quanta, and then convolute with the photoneutron total cross section of the material under study.

\section{Giant Dipole Resonance}
In the photonucleus interaction cross section, the process taking place with the lowest energy is giant dipole resonance. When photons with wavelength comparable to the size of the nucleus, they see the nucleus as a whole. The oscillating electrinc field displaces the protons away from their position and a electric dipole is formed. This is a form of the excitation of the nucleus, and when the nucleus deexcites, one or more neutrons can be released.


\section{quasideutron}
With higher photon energy, the photons start to see the proton-neutron pair in the nucleus.


\section{\texorpdfstring{$\Delta$}{Delta} production}
If the photon energy goes higher, the photons can see individual nucleons and can excite the proton and form $\Delta$ resonances. Protons and neutrons can be excited to form $\Delta$ resonances with the same quark content but higher angular momentum. They decay and generate neutrons.