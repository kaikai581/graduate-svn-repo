\chapter{The RPC Muon Detector}

The development of Resistive Plate Chamber (RPC) was first motivated by the need to improve detectors' timing characteristic. The first improvement was Pestov spark counter developed by Yu.N. Pestov and G.V. Fedotovich. Later on the great simplification of the realization of the concepts of Pestov counter introduced in 1981 by R. Santonico and R. Cardarelli~\cite{Santonico1981} leads to RPC.


\section{RPC bare chambers}
A RPC bare chamber is a parallel plate gas detector for detecting charged particles. A Daya Bay RPC bare chamber is formed by two 2 mm bakelite sheets separated by spacers to form a 2 mm gas gap. A gas combination of Ar, isobutane and R-134A flows in the gap. When charged particles pass through the gas gap, Ar molecules are ionized and the released electrons then undergo acceleration by the applied electric field. This primary electron then ionizes other Ar molecules and generates an electron avalanche. When the avalanche electrons drift the the anode, they induce electric signals which can then be readout. The isobutane can absorb ultraviolet photons and prevents a secondary streamer. The R-134A has high electron affinity and can restrict the size of the streamer.


\section{RPC Modules}
A Daya Bay RPC module is primarily composed of 4 layers of RPC bare chambers. Each layer is formed by 2 RPC bare chambers with the same length but different width. The larger RPC bare chamber has a dimension of 2.1 m $\times$ 1.1 m. The smaller one has a dimension of 2.1 m $\times$ 1.0 m. From bottom to top, the size of the RPC bare chambers alternates so as to offset the dead region due to the edge of the bare chambers. Each RPC layer is equipped with a 2.1 m $\times$ 2.1 m copper-clad FR-4 readout plane consisting of 8 readout strips. The 4 readout layers are oriented, from bottom to top, in the $x$, $y$, $y$, $x$ directions. With this readout arrangement, the position of the incident particle can be reconstructed.


\section{RPC Readout electronics}
Each strip is connected to the Front End Card (FEC).